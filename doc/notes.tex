\documentclass[english,11pt,a4paper]{article}

\usepackage{pgfplots}
\pgfplotsset{compat=newest}
%% the following commands are sometimes needed
\usetikzlibrary{plotmarks}
\usepackage{grffile}

\usepackage{latexsym}
\usepackage{babel}
% %\usepackage{rotating}
\usepackage{subfig}
%\usepackage{multibbl}
%\usepackage{calc}
%\usepackage{lastpage}
\usepackage{amsmath,epsfig,amssymb,amsbsy, amsthm}
%\usepackage{euproposal}
\usepackage{multirow}
%\usepackage{natbib}
%\usepackage{multibib}
%\usepackage{fancybox}
\usepackage{nicefrac}
%\usepackage[table]{xcolor}
%\usepackage[right]{eurosym}



\usepackage[bordercolor=white,backgroundcolor=gray!30,linecolor=black,colorinlistoftodos]{todonotes}

%\usepackage[T1]{fontenc}

\usepackage{wrapfig,graphicx}
\usepackage{url}
%\usepackage{titlesec}

\usepackage{color}
\usepackage{colortbl}


\newcommand{\todobase}[3]{\todo[color=#3,inline]{#2}}

\newcommand{\jl}[1]{\todobase{J}{#1}{yellow}}
\newcommand{\tmo}[1]{\todobase{J}{#1}{pink}}
\newcommand{\mm}[1]{\todobase{J}{#1}{cyan}}

\renewcommand{\baselinestretch}{1.2} 

\newcommand\dimOmega{d}
\newcommand{\ngradients}{m}
\newcommand{\mfindex}{k} %k
\newcommand{\unarycost}{s}
\newcommand{\mfweights}{b}
\newcommand{\bitem}{\begin{itemize}}
\newcommand{\eitem}{\end{itemize}}
\newcommand{\mc}[1]{\mathcal{#1}}
\newcommand{\mb}[1]{\mathbb{#1}}
\newcommand{\R}{\mathbb{R}}
\newcommand{\B}{\mathbb{B}}
\newcommand{\U}{\mathbb{U}}
\newcommand{\E}{\mathbb{E}}
\newcommand{\Q}{\mathbb{Q}}
\newcommand{\Z}{\mathbb{Z}}
\newcommand{\PP}{\mathbb{P}}
\newcommand{\bpm}{\begin{pmatrix}}      
\newcommand{\epm}{\end{pmatrix}}
\newcommand{\T}{\top}
\newcommand{\tr}{\mathsf{tr}}
\newcommand{\ol}[1]{\overline{#1}}      
\newcommand{\la}{\langle}
\newcommand{\ra}{\rangle}
\newcommand{\si}{\sigma}
\newcommand{\mrm}[1]{\mathrm{#1}}
\newcommand{\msf}[1]{\mathsf{#1}}
\newcommand{\Diag}{\mathrsf{Diag}}
\newcommand{\row}[2]{{#1}_{#2,\bullet}}
\newcommand{\col}[2]{{#1}_{\bullet,#2}}
\newcommand{\veps}{\varepsilon}
\newcommand{\toset}{\rightrightarrows}
\newcommand{\w}{\omega}
\newcommand{\gdw}{\Leftrightarrow}
\newcommand{\vphi}{\varphi}
\newcommand{\tmop}[1]{\ensuremath{\operatorname{#1}}}
\newcommand{\norm}[1]{\Vert #1 \Vert}
\newcommand{\normc}[1]{| #1 |}
\newcommand{\bbR}{\mathbb{R}}
\newcommand{\bbRext}{\mathbb{R} \cup \{ \infty \} }
\providecommand{\iprod}[2]{\langle#1,#2\rangle}


\newcommand{\bi}{\begin{itemize}}
\newcommand{\ei}{\end{itemize}}
\newcommand{\HH}{\mathcal{H}}
\newcommand{\MM}{\mathcal{M}}

\newcommand{\DD}{\mathcal{D}}
\newcommand{\JJ}{\mathcal{J}}
\newcommand{\Ss}{\mathcal{S}}

\newcommand{\reps}{\rho_{\veps}}
\newcommand{\geqs}{\geqslant}
\newcommand{\leqs}{\leqslant}


%\newcommand{\cref}[1]{ [{#1}]}
\newcommand{\cref}[1]{ {\tiny[{#1}]}}
%\newcommand{\tm}[1]{[#1] }
\newcommand{\tm}[1]{}

\newcommand{\VI}{\mathsf{VI}}
\newcommand{\SOL}{\mathsf{SOL}}
\newcommand{\graph}{\mathsf{gph}}
\newcommand{\dom}{\mathsf{dom}}
\newcommand{\rint}{\mathsf{rint}}
\newcommand{\bd}{\mathsf{bd}}
\newcommand{\rge}{\mathsf{rge}}
\newcommand{\epi}{\mathsf{epi}}
\newcommand{\lev}{\mathsf{lev}}
\newcommand{\argmin}{\mathsf{argmin}}
\newcommand{\argmax}{\mathsf{argmax}}

\newcommand{\detail}{\ding{43} \structure{Detail}}

\newcommand{\fr}[3][]{\frame{\frametitle{#2}\begin{textblock*}{6cm}(12cm,1.5cm)\rotatebox{90}{\parbox{7cm}{\scriptsize\structure{#1}}}\end{textblock*}\bi #3 \ei}}
\newcommand{\frb}[3][]{\frame{\frametitle{#2}\begin{textblock*}{6cm}(12cm,1.5cm)\rotatebox{90}{\parbox{7cm}{\scriptsize\structure{#1}}}\end{textblock*}#3}}
\newcommand{\tmemtitle}{\textbf}

\newcommand{\mathbbm}{\mathbb}
\newcommand{\C}{\mathcal{C}}
\newcommand{\D}{\mathcal{D}}
\newcommand{\Rn}{\mathbb{R}^n}
\newcommand{\Rm}{\mathbb{R}^m}
\newcommand{\Rmxn}{\mathbb{R}^{m \times n}}
\newcommand{\Rd}{\mathbb{R}^d}
\newcommand{\Rl}{\mathbb{R}^L}
\newcommand{\beq}{\begin{equation}}
\newcommand{\eeq}{\end{equation}}
\newcommand{\beqa}{\begin{eqnarray}}
\newcommand{\eeqa}{\end{eqnarray}}
\newcommand{\bc}{\begin{center}}
\newcommand{\ec}{\end{center}}
\newcommand{\lcb}{\left\{}
\newcommand{\rcb}{\right\}}
\newcommand{\lbr}{\left(}
\newcommand{\rbr}{\right)}
\newcommand{\seq}{\subseteq}
\newcommand{\PC}{\Pi_{\C}}
\newcommand{\PD}{\Pi_{\D}}
\newcommand{\assign}{:=}
\newcommand{\Om}{\Omega}
\newcommand{\BV}{\tmop{BV}}
\newcommand{\SBV}{\tmop{SBV}}
\newcommand{\Ccinf}{C_c^{\infty}}
\newcommand{\Dloc}{\D_{\tmop{loc}}}
\newcommand{\Hausd}{\mathcal{H}}
\newcommand{\tmtextbf}[1]{{\bfseries{#1}}}
\newcommand{\tmtextit}[1]{{\itshape{#1}}}
\newcommand{\tmem}[1]{{\em #1\/}}
%\newcommand{\subs}[1]{\smallskip\noindent\textbf{#1.} }
\newcommand{\subs}[1]{\medskip\noindent\textbf{#1.} }
%\newcommand{\subs}[1]{\noindent\textbf{#1.} }
%\newcommand{\subs}[1]{\paragraph{#1.} }

\newcommand{\wi}{.1\columnwidth}

\newcommand \M   {{\mathcal{M}}}                     % manifold

\newcommand \Sone       {{{\Ss}^1}}                           % S^1
\newcommand \TVS        {{TV_\Sone}}                        % TV_{S^1}
\newcommand \TVM        {{TV_\M}}                        % TV_{S^1}
\newcommand \Stwo       {{{\Ss}^2}}                           % S^1
\newcommand \TVStwo        {{TV_\Stwo}}                        % TV_{S^1}
\newcommand \TVSg       {{TV_\Sone^g}}                      % TV_{S^1}^g
\newcommand \TV         {{TV}}                              % TV
\newcommand{\dx}         {\,dx}
\newcommand \Su         {{S_u}}                             % subset
% of Omega w 
\newcommand \omegaDim   {{m}}                               % image domain dimension
\newcommand \dH         {{d\;\!\mathcal{H}^{\omegaDim-1}}}  % Hausdorff measure
\newcommand{\Div}{\tmop{Div}}


%\arrayrulecolor{black} % green tables otherwise
%\renewcommand{\familydefault}{\sfdefault}
%\usepackage{arial}

\newcommand{\fc}{^{\ast}}
\newcommand{\fcc}{^{\ast\ast}}
\newcommand{\ow}{\text{otherwise}}

\newcommand{\RL}{\R^L}
\newcommand{\Rb}{\bar{\R}}

%\newtheorem{thm}{Theorem}[section]
%\newtheorem{prop}[thm]{Proposition}

\newtheorem{lem}{Lemma}
\newtheorem{prop}{Proposition}
\newtheorem{cor}{Corollary}

\newcommand{\dat}{\boldsymbol{\rho}}
\newcommand{\dats}{\boldsymbol{\sigma}}
\newcommand{\reg}{\boldsymbol{\Phi}}
\newcommand{\Gr}{\mathbf{1}}
\newcommand{\ul}{\boldsymbol{u}}
\newcommand{\vl}{\boldsymbol{v}}
\newcommand{\gl}{\boldsymbol{g}}
\newcommand{\pl}{\boldsymbol{p}}
\newcommand{\ql}{\boldsymbol{q}}
\newcommand{\Tau}{\mathrm{T}}


\usepackage{a4wide}
\usepackage{tikz}

\title{pdsolver Notes}
\usepackage{amsmath,epsfig,amssymb,amsbsy}

\begin{document}
\maketitle

\section{Problem class and scaling}
The class of problems considered within this framework is given as the following:
\begin{equation}
  \underset{x \in \bbR^n, z \in \bbR^m} \min ~ g( \Tau^{\frac{1}{2}} x) + f(\Sigma^{-\frac{1}{2}} z), \quad \text{s.t.} \quad z = \Sigma^{\frac{1}{2}} K \Tau^{\frac{1}{2}} x.
  \label{eq:scaled_problem}
\end{equation}
Here $g : \bbR^n \to \bbRext$ and $f : \bbR^m \to \bbRext$ are convex, proper, closed functions. $K \in \bbR^{m \times n}$ is a matrix and $\Tau \in \bbR^{n \times n}$, $\Sigma \in \bbR^{m \times m}$ are symmetric positive definite scaling (preconditioning) matrices. The dual problem of \eqref{eq:scaled_problem} is given as
\begin{equation}
  \underset{y \in \bbR^m, w \in \bbR^n} \max ~ -g^*(\Tau^{-\frac{1}{2}} w) - f^*(\Sigma^{\frac{1}{2}} y), \quad \text{s.t.} \quad w=-\Tau^{\frac{1}{2}} K^T \Sigma^{\frac{1}{2}} y.
  \label{eq:scaled_dual_problem}
\end{equation}
Note that the scaled problems can be written using the substitutions
\begin{equation}
  \begin{aligned}
    &\widehat x = \Tau^{-\frac{1}{2}} x, \\
    &\widehat z = \Sigma^{\frac{1}{2}} z, \\
    &\widehat y = \Sigma^{-\frac{1}{2}} y, \\
    &\widehat w = \Tau^{\frac{1}{2}} w, \\
    &\widehat g = g \circ \Tau^{\frac{1}{2}}, \\
    &\widehat f = f \circ \Sigma^{-\frac{1}{2}}, \\
    &\widehat K = \Sigma^{\frac{1}{2}} K \Tau^{\frac{1}{2}}.
  \end{aligned}
\end{equation}

\section{Backends}
\subsection{Primal-Dual Hybrid Gradient Method}
The algorithm is given as:
\begin{equation}
  \begin{aligned}
    &x^{k+1} = (I + \tau \Tau \partial g)^{-1}(x^k - \tau \Tau K^T y^k),\\
    &y^{k+1} = (I + \sigma \Sigma \partial f^*)^{-1}(y^k + \sigma \Sigma K (x^{k+1} + \theta (x^{k+1} - x^k))).\\
  \end{aligned}
  \label{eq:pdhgm}
\end{equation}
The additional variables $w$ and $z$ are given as the following:
\begin{equation}
  \begin{aligned}
    &w^{k+1} = \frac{\Tau^{-1}(x^k - x^{k+1})}{\tau} - K^T y^k,\\
    &z^{k+1} = \frac{\Sigma^{-1}(y^k - y^{k+1})}{\sigma} + K (x^{k+1} + \theta (x^{k+1} - x^k)).
  \end{aligned}
  \label{eq:pdhgm_zw}
\end{equation}

The method produces a sequence of iterates
$(x^k, z^k, y^k, w^k)$ converging to a solution of the (unscaled) primal and dual problem. As a stopping criterion, we consider the 
primal and dual residuals of the scaled problem:
\begin{equation}
  r_p^{k+1} = \norm{\widehat K \widehat x^{k+1} - \widehat z^{k+1}} = \norm{\Sigma^{\frac{1}{2}} (K x^{k+1} - z^{k+1})} = \norm{Kx^{k+1} - z^{k+1}}_{\Sigma},
\end{equation}
\begin{equation}
  r_d^{k+1} = \norm{\widehat K^T \widehat y^{k+1} + \widehat w^{k+1}} = \norm{\Tau^{\frac{1}{2}} (K^T y^{k+1} + w^{k+1})} = \norm{K^T y^{k+1} + w^{k+1}}_{\Tau}.
\end{equation}
In particular, the algorithm is stopped if $r_p < \varepsilon_a + \varepsilon_r \norm{z^{k+1}}_{\Sigma}$ and $r_d < \varepsilon_a + \varepsilon_r \norm{w^{k+1}}_{\Tau}$ for some $\varepsilon_a, \varepsilon_r > 0$.

\end{document}
